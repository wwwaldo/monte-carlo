\documentclass{beamer}
\newcommand\di{\partial}
\newcommand\pars[1]{\left(#1\right)}
\newcommand\mat[1]{#1}
\usepackage{amsmath,amssymb,amsthm,bbm,bm, amsfonts}
\usepackage{algorithm}
\usepackage{algpseudocode}
\setbeamertemplate{footline}[frame number]
\title{Solving Laplace's Problem Via Random Walks}
\author{Caroline Lin and Dmitry Paramonov}
\date{December 4th, 2018}

\begin{document}
\begin{frame}
  \titlepage%
\end{frame}
% Slide Outline:
% Laplace's Problem
% Poisson's Problem (A Generalization)
\begin{frame}{Laplace's Problem}
\begin{itemize}
\item We wish to solve \emph{Laplace's Problem}.
Here, we have some closed domain $\Omega$ with a boundary $\di\Omega$.
We are given a function $g\pars{\vec{x}}$, defined on $\vec{x}\in\di\Omega$.
\item We wish to find a function $u\pars{\vec{x}}$, defined on $\Omega$,
such that $\Delta u=0$ on $\Omega$
and such that $u\pars{\vec{x}}=g\pars{\vec{x}}$ on $\di\Omega$.
\item A generalization to this is \emph{Poisson's Problem},
wherein we also have some $f\pars{\vec{x}}$ defined on $\vec{x}\in\Omega$,
and where we require that $\Delta u\pars{\vec{x}}=f\pars{\vec{x}}$ on $\Omega$.
\end{itemize}
\end{frame}

% Possible Approaches (Finite Whatever)
\begin{frame}{Finite Differences}
One common approach: solve using the 2D discrete Laplacian.
	\[ L = \frac{1}{h^2} \begin{bmatrix}
		0 & 1 & 0 \\
		1 & -4& 1 \\
		0 & 1 & 0
	\end{bmatrix} \]
\begin{itemize}
\item Pros: solving becomes a matrix solve
\item Pros: Boundary conditions are encoded into a vector
\item Cons: Needs a Cartesian grid
\end{itemize}

Alternatives: Finite elements, finite volumes.
\end{frame}
% Random Walks
\begin{frame}{Random Walks in a nutshell}
A ``normally distributed'' stochastic process in time, i.e.

$E(B(t)) \sim N(0, t)$ for all times $t \geq 0$
\end{frame}

\begin{frame}{Laplace's Equation with Random Walks}
Ito's formula:
$u(x) = E^x[ g(\mathbf{B}(T))] + \frac{1}{2} E^x \left[ \int_0^T f(B(s))ds \right]$

For Laplace's, take the mean over the boundary values, weighted by frequency of random walks.
\end{frame}
% Why It Works (In Theory)

% Extensions: Interior Finite Differences
\begin{frame}{Extension: Interior Finite Differences}
\begin{itemize}
\item However, random walks are slow.
% \item However, using only random walks can be very slow and inefficient.
% Having to do many random walks for each point we can about can be very slow.
\item Instead, define a Cartesian grid within the main boundary.
\item Can use finite differences on the inside of this grid.
\item The boundary values can be found using random walks.
\item Amount of random walks now scales with $h^{-1}$, instead of $h^{-2}$.
\end{itemize}
\end{frame}

% Extensions: Self-Contacting Boundary Coupling
\begin{frame}{Extensions: Self-Contacting Boundary Coupling}
\begin{itemize}
\item Furthermore, we can add coupling to the grid points.
\item If a random walk touches the interior boundary,
then the expected value of that random walk
is the expected value of a random walk starting at this new point.
\item We thus store an extra matrix $\mat{F}$.
$F_{i,j}$ is the number of random walks starting at boundary point $i$
and ending at boundary point $j$.
If $K$ is the number of random walks performed
and $\vec{b}$ is the vector of total values obtain by random walks,
we solve
$$\vec{u}=\frac{1}{K}\mat{F}\vec{u}+\frac{1}{K}\vec{b},$$
\end{itemize}
\end{frame}

% Extensions: Walk On Spheres
\begin{frame}{Extensions: Walk On Spheres}
\begin{itemize}
\item Random walks can be slow.
\item But for Laplace's Problem, only the end result matters.
\item Points on a large sphere, uniformly distributed,
are distributed the same as a random walk until the boundary.
\item Thus, we can just make large steps in a uniformly random direction.
\end{itemize}
\end{frame}
% Results: Accuracy
% Results: Spiky Domains
% Results: Runtime
% Conclusions
\end{document}
